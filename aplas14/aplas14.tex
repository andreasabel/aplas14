\nonstopmode

% type-set with agda --latex
\newcommand{\AGDALATEX}[1]{#1}
\newcommand{\PLAINLATEX}[1]{}

\newcommand{\PDFLATEX}[1]{#1}

% For having two versions of a paper in one file.
% Stuff that does not fit into the short version can be encosed in \LONGVERSION{...}
\ifdefined\LONGVERSION
  \relax
\else
% short version:
\newcommand{\LONGVERSION}[1]{}
\newcommand{\SHORTVERSION}[1]{#1}
% % long version:
% \newcommand{\LONGVERSION}[1]{#1}
% \newcommand{\SHORTVERSION}[1]{}
% \newcommand{\SHORTVERSION}[1]{BEGIN~SHORT\ #1 \ END~SHORT}
\fi
\newcommand{\LONGSHORT}[2]{\LONGVERSION{#1}\SHORTVERSION{#2}}



\documentclass{llncs}


%\usepackage{times}
\usepackage{mathptmx}
%\usepackage{amsfonts,amssymb,textgreek,stmaryrd}
\PDFLATEX{
\usepackage[postscript]{ucs}
%\usepackage[postscript,combine]{ucs} %% combine destroys all unicode
\usepackage[utf8x]{inputenc}
}
\usepackage{pifont}
\usepackage{natbib}
\renewcommand{\bibname}{\refname}
% \usepackage{unicode-math}  % xelatex only, but not needed

% Agda stuff

\AGDALATEX{%
\usepackage{latex/agda}
% \usepackage[conor]{latex/agda} %% almost Barbie
}
\PLAINLATEX{%
\usepackage{verbatim}
\newenvironment{code}{\verbatim}{\endverbatim}
\long\def\AgdaHide#1{} % Used to hide code from LaTeX.
}
\DeclareMathAlphabet      {\mathbfit}{OML}{cmm}{b}{it}
\DeclareUnicodeCharacter{8718}{$\blacksquare$}
\DeclareUnicodeCharacter{8759}{::}
\DeclareUnicodeCharacter{9656}{$\blacktriangleright$}
\DeclareUnicodeCharacter{9733}{$\star$}
\DeclareUnicodeCharacter{10214}{$\llbracket$}
\DeclareUnicodeCharacter{10215}{$\rrbracket$}
\DeclareUnicodeCharacter{10218}{$\langle\!\langle$} % hex 27EA ⟪
\DeclareUnicodeCharacter{10219}{$\rangle\!\rangle$}
\DeclareUnicodeCharacter{119916}{$\mathbfit{E}$}
\DeclareUnicodeCharacter{119957}{$\mathbfit{t}$}
\DeclareUnicodeCharacter{119958}{$\mathbfit{u}$}
\DeclareUnicodeCharacter{119951}{$\mathbfit{n}$}
\DeclareUnicodeCharacter{119912}{$\mathbfit{A}$}
\DeclareUnicodeCharacter{120016}{$\mathcal{A}$} %% \mathscr didn't do anything
\DeclareUnicodeCharacter{120017}{$\mathcal{B}$}
\DeclareUnicodeCharacter{10214}{$[\![$}
\DeclareUnicodeCharacter{10215}{$]\!]$}
\DeclareUnicodeCharacter{916}{$\Delta$}

\begin{document}
%
\frontmatter          % for the preliminaries
%
\pagestyle{headings}  % switches on printing of running heads
%
\mainmatter              % start of the contributions
%
\title{A Formalized Proof of Strong Normalization for Guarded Recursive Types}
%
\titlerunning{Strong Normalization for Guarded Types}  % abbreviated title (for running head)
%                                     also used for the TOC unless
%                                     \toctitle is used
%
\author{
Andreas Abel\inst{1}
\and
Andrea Vezzosi\inst{1}
%\thanks{Supported by 
%Vetenskapsr\aa{}det framework grant to the ProgLog group (Thierry Coquand et al.) of
%the Department of Computer Science and Engineering at Chalmers and
%Gothenburg University} 
\and 
Lars Birkedal\inst{2} 
\and 
Ranald Clouston\inst{2}
\and
Hans Bugge Grathwohl\inst{2}
}
%
\authorrunning{Abel, Vezzosi, Birkedal, Clouston, Grathwohl} % abbreviated author list (for running head)
%
%%%% list of authors for the TOC (use if author list has to be modified)
%\tocauthor{Ivar Ekeland, Roger Temam, Jeffrey Dean, David Grove,
%Craig Chambers, Kim B. Bruce, and Elisa Bertino}
%
\institute{
Computer Science and Engineering, Chalmers and Gothenburg University,\\
R\"annv\"agen 6, 41296 G\"oteborg, Sweden,
\email{andreas.abel@gu.se,vezzosi@chalmers.se}
\and 
Department of Computer Science, Aarhus University, Denmark\\
\email{\{birkedal,hbugge,ranald.clouston\}@cs.au.dk}
}


\maketitle              % typeset the title of the contribution

%%% Abstract
\begin{abstract}
\input{abstract.txt}  
\end{abstract}

% Template
\AgdaHide{
\begin{code}
  
\end{code}
}


\section{Introduction}
\label{sec:intro}

In untyped lambda calculus, fixed-point combinators can be defined
using self-app\-li\-ca\-tion.  Such combinators can be assigned recursive
types, albeit only negative ones.  Since such types introduce logical
inconsistency, they are problematic, especially in Martin-L\"of
Type Theory and other systems based on the Curry-Howard isomorphism.
\citet{nakano:lics00} introduced \emph{a modality for recursion} that
allows a stratification of negative recursive types to recover
consistency.  In essence, each negative recursive occurrence needs to
be \emph{guarded} by the modality; this coined the term \emph{guarded
  recursive types} \citep{birkedalMogelberg:lics13}.\footnote{Not to
  be confused with \emph{Guarded Recursive Datatype Constructors}
  \citep{xiChenChen:popl03}.} 
Nakano's modality has found applications in functional reactive
programming \citep{krishnaswamiBenton:lics11} where it is referred to
as \emph{later} modality.


%%% Local Variables: 
%%% mode: latex
%%% TeX-master: "aplas14.tex"
%%% End: 



\section{Formalized Syntax}
\label{sec:syntax}

In this section, we discuss the formalization of types, terms, and
typing of $\lambdalater$ in Agda.  It will be necessary to talk about
meta-level types, \ie, Agda's types, thus, we will refer to
$\lambdalater$'s type constructors as $\hattimes$, $\hatto$,
$\hatlater$, and $\hatmu$.

\subsection{Types Represented Coinductively}

Instead of representing fixed-points as syntactic construction on
types, which would require a non-trivial equality on types induced by
$\hatmu X A = \subst{\hatmu X A}X A$, we use \emph{meta-level} fixed-points,
\ie, Agda's recursion mechanism.  Extensionally, we are implementing
\emph{infinite type expressions} over the constructors $\hattimes$,
$\hatto$, and $\hatlater$.  The guard condition on recursive types is then
becoming an instance of Agda's ``guard condition'', \ie, the condition
the termination checker imposes on recursive programs.

Viewed as infinite expressions, guarded types are regular trees with
an infinite number of $\hatlater$-nodes on each infinite path.  This
can be expressed as the mixed coinductive-inductive (meta-level) type
\[
  \nu X \mu Y.\; (Y \times Y) + (Y \times Y) + X. 
\]
The first summand stands for the binary constructor $\hattimes$, the
second for $\hatto$, and the third for the unary $\hatlater$.  The
nesting of a least-fixed point ($\mu$) inside a greatest fixed-point
($\nu$) ensures that on each path, we can only take alternatives
$\hattimes$ and $\hatto$ a finite number of times before we have to
choose the third alternative $\hatlater$ and restart the process.


\input{InfiniteTypes}

\input{Terms}


%%% Local Variables: 
%%% mode: latex
%%% TeX-master: "aplas14.tex"
%%% End: 

% Syntax of types
% Explanation of Agda's coinduction

% Contexts
% Variables
% Terms


\subsection{Examples}
\label{sec:examples}

\input{AgdaExamples}

%%% Local Variables: 
%%% mode: latex
%%% TeX-master: "aplas14.tex"
%%% End: 

% Abbreviations, Cast
% Examples: 
% * Y (in Agda)
% * Streams & cons
% More examples in math notation (Stream map)

\input{Substitution}
% Renaming & Substitution (inspired by McBride)
% Lemmata (without proof)
% Equivalence on substitutions
% Lemmata: id, assoc (composition)

\section{Strong Normalization}
\label{sec:sn}

\input{sn}
\input{NReduction}
%%% Local Variables: 
%%% mode: latex
%%% TeX-master: "aplas14.tex"
%%% End: 

% sn (acc), beta-reduction

% (sized) SN, strong head reduction
% closure under SNe substitution
% other closure properties

% SN closed under "anti-rename"
% needs: Inductive Renaming
% soundness and completeness (in prose)

\section{Soundness}
\label{sec:soundess}

A well-established technique 
\citep{tait:functionalsFiniteTypeI} to prove strong normalization is to model
each type $\va$ as a set $\A = \den{\va}$ of sn terms.  Each so-called
semantic type $\A$
should contain the variables in order to interpret open terms by
themselves (using the identity valuation).
To establish the conditions of semantic types compositionally, 
the set $\A$ needs to be \emph{saturated},
\ie, contain $\SNe$ (rather than just the variables) and be closed
under strong head expansion (to entertain introductions).

\input{SAT}

\input{Soundness}

%%% Local Variables: 
%%% mode: latex
%%% TeX-master: "aplas14.tex"
%%% End: 

% SAT3:
% TermSet (raw candidates)
% Kripke-SAT (SNe <= A <= SN, closure under expansion, monotonicity)

\section{\AgdaDatatype{SN} correctness}
\label{sec:SNc}

\input{SNtosn}

%%% Local Variables: 
%%% mode: latex
%%% TeX-master: "aplas14.tex"
%%% End: 


% converting from SN to sn.


\section{Conclusions}
\label{sec:concl}

In this paper, we presented a family of reduction
relations for simply-typed lambda calculus with Nakano's modality for
recursion.  While each of the relations is strongly normalizing, their
limit, which they approximate, is the diverging, unrestricted
reduction relation.  A similar result, weak normalization, has been obtained by
\cite{krishnaswamiBenton:icfp11}
using hereditary substitutions, albeit
for a system without recursive types.

Strong normalization has formally been proven in Agda using a
saturated sets semantics based on an inductive notion of strong
normalization.  Herein, we represented recursive types as infinite
type expressions and terms as intrinsically well-typed ones.

Our treatment of infinite type expressions was greatly simplified by
adding an extensionality axiom for the underlying coinductive type to
Agda's type theory.  This would not have been necessary in a more
extensional theory such as \emph{Observational Type Theory}
\citep{altenkirchMcBrideSwierstra:plpv07} as shown in \citep{mcBride:calco09}.  Possibly \emph{Homotopy Type
Theory} \citep{hott}
would also address this problem, but there the status of
coinductive types is yet unclear.

For the future, we would like to investigate how to incorporate
guarded recursive types into a dependently-typed language, and how
they relate to other approaches like coinduction with sized
types, for instance.

\paradot{Acknowledgments}
Thanks to Lars Birkedal, Ranald Clouston, and Rasmus M\o{}gelberg for
fruitful discussions on guarded recursive types, and Hans Bugge
Grathwohl and Fabien Renaud for useful feedback on the Agda
development and a draft version of this paper.
The first author acknowledges support by Vetenskapsr\aa{}det framework
grant 254820104 (Thierry Coquand).  This paper has been prepared with
Stevan Andjelkovic's Agda-to-LaTeX converter.


%%% Local Variables:
%%% mode: latex
%%% TeX-master: "aplas14.tex"
%%% End:



\bibliographystyle{splncsnat}
\bibliography{auto-aplas14}

\end{document}
