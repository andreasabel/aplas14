\nonstopmode
\documentclass[t]{beamer}
\usepackage{mathptmx}
\usepackage{amsfonts,amssymb,textgreek,stmaryrd}
\usepackage[postscript]{ucs}
%\usepackage[postscript,combine]{ucs} %% combine destroys all unicode
\usepackage[utf8x]{inputenc}
\usepackage{pifont}
%%\usepackage{environ}
\usepackage{latex/agda}

%\documentclass{article}\usepackage{beamerarticle}

\makeatletter
%\def\leqn{\tagsleft@true}
%\def\reqn{\tagsleft@false}
\def\fleq{\@fleqntrue\let\mathindent\@mathmargin \@mathmargin=\leftmargini}
\def\cneq{\@fleqnfalse}
\g@addto@macro{\endsubequations}{\addtocounter{equation}{-1}}
\makeatother

\mode<presentation>
{
  % \usetheme{AnnArbor}
  \useinnertheme[shadow=true]{rounded}
  \useoutertheme{infolines}
  %\setbeamercolor*{title in head/foot}{parent=palette secondary}
  %\useoutertheme{shadow}
  \usecolortheme{wolverine}

  \setbeamerfont{block title}{size={}}
  \setbeamercolor{titlelike}{parent=structure,bg=yellow!85!orange}

  % My modification: centered title with white background
  \setbeamercolor{frametitle}{bg=white}
  \setbeamertemplate{frametitle}{
    \begin{center}
      \Large\insertframetitle
      \par
    \end{center}
  }
  % plus red emphasis
  \setbeamercolor{alerted text}{fg=red}

  % More space between items
  %\defbeamertemplate*{itemize/enumerate body begin}{default}{\itemsep1ex}


  \setbeamercolor{math text}{parent=titlelike}
%  \setbeamercolor{math text displayed}{parent=palette primary}

  % Fix background of theorems/proof
  \setbeamercolor{block body}{parent=palette primary}%{bg=yellow!85!orange}
  \setbeamercolor{block title}{parent=palette secondary}%{bg=orange}



  \setbeamercovered{transparent}
  % or whatever (possibly just delete it)

} % end mode presentation


\usepackage[english]{babel}
\usepackage[utf8x]{inputenc}
%\usepackage{times}  % DYSFUNCTIONAL! looks awful when mixing mathtt and ordinary math.
\usepackage{ifthen}
%\usepackage[T1]{fontenc} %no effect
% Or whatever. Note that the encoding and the font should match. If T1
% does not look nice, try deleting the line with the fontenc.
\usepackage{amsmath}
\usepackage{amssymb}
\usepackage{stmaryrd}
%\usepackage{alltt}
%\usepackage[normalem]{ulem} % strikethrough \sout
%\usepackage{cancel} % strikethrough math mode \cancel
%\usepackage{enumitem} % set label in itemize
%\usepackage{pifont} % for \tickNo
%\usepackage{listings}
%\usepackage[all]{xy}
%\usepackage{proof}
%\usepackage{eurosym}
%\usepackage{graphics}
%\usepackage{bibentry}
%\nobibliography{short}
%\bibliographystyle{plain}
%\input{prooftree}

\DeclareMathSymbol{\chkmark}{\mathord}{AMSa}{"58}

% RGB colors
\definecolor{darkred}{rgb}{0.5,0,0}
\definecolor{darkgreen}{rgb}{0,0.5,0}
\definecolor{darkblue}{rgb}{0,0,0.5}
\definecolor{dirtyred}{rgb}{0.7,0.2,0.1}
\definecolor{dirtygreen}{rgb}{0.2,0.4,0.1}
\definecolor{darkdirtygreen}{rgb}{0.13,0.25,0.07}
\definecolor{dirtyblue}{rgb}{0.07,0.2,0.5}
\definecolor{darkdirtyblue}{rgb}{0.1,0.15,0.35}
\definecolor{lightblue}{rgb}{0.5,0.5,1}
\definecolor{olivegreen}{rgb}{0.5,0.5,0}
\definecolor{brown}{rgb}{0.65,0.35,0} % almost gold
\definecolor{grey}{rgb}{0.33,0.33,0.33}
\definecolor{darkbrown}{rgb}{0.35,0.15,0}
\definecolor{darkgrey}{rgb}{0.16,0.16,0.16}

\newcommand{\textred}[1]{{\color{dirtyred}\textbf{#1}}}

%\newcommand{\tickYes}{{\color{darkgreen}\checkmark}} % does not work
  %since checkmark invokes math mode
%\newcommand{\tickYes}{\ensuremath{\color{darkgreen}\chkmark}}
%\newcommand{\tickNo}{{\color{dirtyred}\hspace{1pt}\ding{55}}}

%\usepackage[curve,matrix,arrow]{xy}
%\usepackage{tikz} % Drawing diagrams
%\usepackage{pgflibraryshapes} %Ellipses


% black text in mbox
\newcommand{\mybox}[1]{\mbox{\color{black}#1}}

% MACROS:
\newcommand{\inst}{}
% evergreens
\newcommand{\bla}{\ensuremath{\mbox{$$}}} % invisible, but not ignored
\newcommand{\der}{\,\vdash}
\newcommand{\of}{{:}}
\newcommand{\red}{\longrightarrow}
\newcommand{\contr}{\mapsto}
\newcommand{\restrict}{\upharpoonright}
\newcommand{\FV}{\ensuremath{\mathsf{FV}}}
\newcommand{\NN}{\mathbb{N}}
\newcommand{\den}[1]{\llbracket #1 \rrbracket}
\newcommand{\Den}[2]{\llbracket #1 \rrbracket_{#2}}
\newcommand{\vect}{\mathaccent"017E }

% latin etc. abbrev
\newcommand{\abbrev}[1]{#1} % alternative: \emph{#1}
\newcommand{\cf}{\abbrev{cf.}\ }
\newcommand{\eg}{\abbrev{e.\,g.}}
\newcommand{\Eg}{\abbrev{E.\,g.}}
\newcommand{\ie}{\abbrev{i.\,e.}}
\newcommand{\Ie}{\abbrev{I.\,e.}}
\newcommand{\etal}{\abbrev{et.\,al.}}
\newcommand{\wwlog}{w.\,l.\,o.\,g.} % \wlog is ``write into log file''
\newcommand{\Wlog}{W.\,l.\,o.\,g.}
\newcommand{\wrt}{w.\,r.\,t.}

% paragraphs
\newcommand{\para}[1]{\paragraph*{\it#1}}
\newcommand{\paradot}[1]{\para{#1.}}

% proof by cases
\newenvironment{caselist}{%
  \begin{list}{{\it Case}}{\setlength{\leftmargin}{5ex}}%
}{\end{list}%
}
\newenvironment{subcaselist}{%
  \begin{list}{{\it Subcase}}{}%
}{\end{list}%
}

\newcommand{\nextcase}{\item~}
% meta-level logic
\newcommand{\mfor}{\ \mbox{for}\ }
\newcommand{\mthen}{\ \mbox{then}\ }
\newcommand{\mif}{\ \mbox{if}\ }
\newcommand{\miff}{\ \mbox{iff}\ }
\newcommand{\motherwise}{\ \mbox{otherwise}}
\newcommand{\mnot}{\mbox{not}\ }
\newcommand{\mand}{\ \mbox{and}\ }
\newcommand{\mor}{\ \mbox{or}\ }
\newcommand{\mimplies}{\ \mbox{implies}\ }
\newcommand{\mimply}{\ \mbox{imply}\ }
\newcommand{\mforall}{\ \mbox{for all}\ }
\newcommand{\mforsome}{\ \mbox{for some}\ }
\newcommand{\mexists}{\mbox{exists}\ }
\newcommand{\mexist}{\mbox{exist}\ }
\newcommand{\mtrue}{\mbox{true}}
\newcommand{\mwhere}{\ \mbox{where}\ }
\newcommand{\mwith}{\ \mbox{with}\ }
\newcommand{\mholds}{\ \mbox{holds}\ }
\newcommand{\rin}{\ \mbox{in}\ }
% proofs
\newcommand{\msince}{\mbox{since}\ }
\newcommand{\mdef}{\mbox{by def.}}
\newcommand{\mass}{\mbox{assumption}}
\newcommand{\mhyp}{\mbox{by hyp.}}
\newcommand{\mlemma}[1]{\mbox{by Lemma~#1}}
\newcommand{\mih}[1][]{\mbox{by ind.hyp.}#1}
\newcommand{\mgoal}[1][]{\mbox{goal\ifthenelse{\equal{#1}{}}{}{~#1}}}
\newcommand{\mby}[1]{\mbox{by #1}}
\newcommand{\minfrule}{\mbox{by inference rule}}


% Inference rules
\newcommand{\rulename}[1]{\ensuremath{\mbox{\sc#1}}}
\newcommand{\ru}[2]{\dfrac{\begin{array}[b]{@{}c@{}} #1 \end{array}}{#2}}
\newcommand{\rux}[3]{\ru{#1}{#2}\ #3}
\newcommand{\nru}[3]{#1\ \ru{#2}{#3}}
\newcommand{\nrux}[4]{#1\ \ru{#2}{#3}\ #4}
\newcommand{\dstack}[2]{\begin{array}[b]{c}#1\\#2\end{array}}
\newcommand{\ndru}[4]{#1\ \ru{\dstack{#2}{#3}}{#4}}
\newcommand{\ndrux}[5]{#1\ \ru{\dstack{#2}{#3}}{#4}\ #5}

% Substitution and function update
% read ``\subst F X A'' as ``substitute F for X in A''
%\newcommand{\subst}[3]{#3[#2 := #1]}
%\newcommand{\subst}[3]{[#1/#2]#3}
\newcommand{\subst}[3]{#3[#1/#2]}
% read ``\update \theta X \G'' as update \theta at point X by \G
\newcommand{\update}[3]{#1[#2 \mapsto #3]}
%\newcommand{\update}[3]{#1,#2 \is #3}

% The calculus
\newcommand{\lambdalater}{\ensuremath{\lambda^{\tlater}}}

% Types
\newcommand{\ta}{\ensuremath{\AgdaBound{a}}}
\newcommand{\tb}{\ensuremath{\AgdaBound{b}}}
\newcommand{\tc}{\ensuremath{\AgdaBound{c}}}
\newcommand{\later}{\mathord{\blacktriangleright}\,}
\newcommand{\tlater}{\mathord{\blacktriangleright}}
%\newcommand{\sem}[1]{\mathrel{\den{#1}}}
\newcommand{\semto}{\mathrel{\den{\to}}}
\newcommand{\semtimes}{\mathrel{\den{\times}}}
\newcommand{\semlater}{\mathop{\den{\tlater}}}
\newcommand{\semtop}{\mathop{\den{\top}}}

\newcommand{\A}{\mathcal{A}}
\newcommand{\B}{\mathcal{B}}
\newcommand{\C}{\mathcal{C}}
\newcommand{\tsize}{\mathsf{size}}

\newcommand{\hatted}[1]{\AgdaInductiveConstructor{\ensuremath{\hat{#1}}}}
\newcommand{\hatto}{\hatted{\to}}
\newcommand{\hattimes}{\hatted{\times}}
\newcommand{\hatlater}{\hatted{\later}}
\newcommand{\hatmu}{\hatted{\mu}}

% Terms
\newcommand{\tfst}{\mathsf{fst}}
\newcommand{\tsnd}{\mathsf{snd}}
\newcommand{\tnext}{\mathsf{next}}
\newcommand{\fst}[1]{\tfst\;#1}
\newcommand{\snd}[1]{\tsnd\;#1}
\newcommand{\pure}[1]{\tnext\;#1}
\newcommand{\apply}[2]{#1 \mathop{∗} #2}
\newcommand{\applyp}[2]{\apply{(#1)}{(#2)}}
\newcommand{\tfix}{\mathsf{fix}}
\newcommand{\afix}{\AgdaFunction{fix}}

% Agda 
\newcommand{\Ty}{\AgdaDatatype{Ty}}
\newcommand{\infTy}{\AgdaRecord{∞Ty}}
\newcommand{\va}{\AgdaBound{a}}
\newcommand{\vb}{\AgdaBound{b}}
\newcommand{\vc}{\AgdaBound{c}}
\newcommand{\vainf}{\AgdaBound{a∞}}
\newcommand{\vbinf}{\AgdaBound{b∞}}
\newcommand{\vcinf}{\AgdaBound{c∞}}
\newcommand{\vi}{\AgdaBound{i}}
\newcommand{\vj}{\AgdaBound{j}}
\newcommand{\vx}{\AgdaBound{x}}
\newcommand{\vy}{\AgdaBound{y}}
\newcommand{\vn}{\AgdaBound{n}}
\newcommand{\vm}{\AgdaBound{m}}
\newcommand{\vt}{\AgdaBound{t}}
\newcommand{\vtprime}{\AgdaBound{t'}}
\newcommand{\vnprime}{\AgdaBound{n'}}
\newcommand{\vu}{\AgdaBound{u}}
\newcommand{\tdelay}{\AgdaCoinductiveConstructor{delay}}
\newcommand{\tforce}{\AgdaField{force}}
\newcommand{\tcoinductive}{\AgdaKeyword{coinductive}}
\newcommand{\ttop}{\AgdaFunction{top}}
\newcommand{\tcast}{\AgdaFunction{cast}}
\newcommand{\Var}{\AgdaDatatype{Var}}
\newcommand{\Gam}{\AgdaBound{Γ}}
\newcommand{\Del}{\AgdaBound{Δ}}
\newcommand{\Cxt}{\AgdaDatatype{Cxt}}
\newcommand{\ECxt}{\AgdaDatatype{ECxt}}
\newcommand{\Ehole}{\AgdaDatatype{Ehole}}
\newcommand{\PCxt}{\AgdaDatatype{PCxt}}
\newcommand{\PNe}{\AgdaDatatype{PNe}}
\newcommand{\Tm}{\AgdaDatatype{Tm}}
\newcommand{\tzero}{\AgdaInductiveConstructor{zero}}
\newcommand{\tsuc}{\AgdaInductiveConstructor{suc}}
\newcommand{\tvar}{\AgdaInductiveConstructor{var}}
\newcommand{\tabs}{\AgdaInductiveConstructor{abs}}
\newcommand{\tapp}{\AgdaInductiveConstructor{app}}
\newcommand{\tpair}{\AgdaInductiveConstructor{pair}}
\newcommand{\afst}{\AgdaInductiveConstructor{fst}}
\newcommand{\asnd}{\AgdaInductiveConstructor{snd}}
\newcommand{\anext}{\AgdaInductiveConstructor{next}}
\newcommand{\arename}{\AgdaFunction{rename}}
\newcommand{\tmapS}{\AgdaFunction{mapS}}
\newcommand{\ttail}{\AgdaFunction{tail}}
\newcommand{\vmapS}{\mathit{mapS}}
%\newcommand{\t}{\AgdaInductiveConstructor{}}
%\newcommand{\t}{\AgdaInductiveConstructor{}}
\newcommand{\aleq}{\ensuremath{\mathrel{\AgdaFunction{≤}}}}
\newcommand{\propeq}{\AgdaDatatype{≡}}
\newcommand{\bisim}{\AgdaDatatype{≅}}
\newcommand{\bisiminf}{\AgdaDatatype{∞≅}}
\newcommand{\sn}{\AgdaDatatype{sn}}
\newcommand{\SN}{\AgdaDatatype{SN}}
\newcommand{\SNe}{\AgdaDatatype{SNe}}
\newcommand{\nred}[1]{\ensuremath{\mathrel{\AgdaDatatype{⟨}\AgdaBound{#1}\AgdaDatatype{⟩⇒β}}}}
\newcommand{\nwhr}[1]{\ensuremath{\mathrel{\AgdaDatatype{⟨}\AgdaBound{#1}\AgdaDatatype{⟩⇒}}}}
\newcommand{\IndRen}{\AgdaDatatype{IndRen}}
\newcommand{\tcong}{\AgdaInductiveConstructor{cong}}
\newcommand{\texp}{\AgdaInductiveConstructor{exp}}
\newcommand{\vP}{\AgdaBound{P}}
\newcommand{\vE}{\AgdaBound{E}}
\newcommand{\vEt}{\AgdaBound{Et}}
\newcommand{\vEtprime}{\AgdaBound{Et'}}
\newcommand{\vC}{\AgdaBound{C}}
\newcommand{\vCt}{\AgdaBound{Ct}}
\newcommand{\vCtprime}{\AgdaBound{Ct'}}
\newcommand{\NbetaCxt}{\AgdaDatatype{NβCxt}}
\newcommand{\NbetaHole}[3]{\ensuremath{\mathrel{\AgdaBound{#1}\,\AgdaDatatype{≡}\,\AgdaBound{#2}\,\AgdaDatatype{[}\AgdaBound{#3}\AgdaDatatype{]}}}}
\newcommand{\whr}{\ensuremath{\mathrel{\AgdaFunction{⇒}}}}
\newcommand{\SizeLt}{\AgdaFunction{Size<}}
\newcommand{\asize}{\AgdaFunction{size}}

% Generic shorthands
\newcommand{\aic}[1]{\AgdaInductiveConstructor{#1}}
\newcommand{\af}[1]{\AgdaFunction{#1}}
\newcommand{\ab}[1]{\AgdaBound{#1}}

%%% Local Variables: 
%%% mode: latex
%%% TeX-master: "aplas14.tex"
%%% End: 


\newcommand{\nex}{\tnext\,}
\newcommand{\Y}{\mathsf{Y}}
\newcommand{\contrSN}{\contr^\SN}
\newcommand{\redSN}{\red^\SN}
\newcommand{\clos}[1]{\overline{#1}}
\renewcommand{\sn}{\mathsf{sn}}
\renewcommand{\SN}{\mathsf{SN}}
\newcommand{\SAT}{\mathsf{SAT}}
\newcommand{\slater}[1]{\tlater\!_{#1}\,}
\newcommand{\tid}{\mathsf{id}}

% \renewcommand*\ttdefault{txtt} % for listing package

% \newcommand{\defHaskelllistings}{%
%   \lstset{%
%     language=Haskell,%
%     basicstyle=\ttfamily\small\color{darkdirtyblue},% \ttfamily
%     keywordstyle=\ttfamily\bfseries,% \underbar
%     identifierstyle=,%
%     commentstyle=\itshape,%
%     columns=flexible,%spaceflexible,% fixed,% flexible,%
%     showstringspaces=false,%
% %    xleftmargin=\codeindent,% defined below
%     breaklines=true,%
%     deletekeywords={succ,zero,head,tail,zipWith,Either,List},%
%     morekeywords={Set,Size,fun,cofun,pattern},% ,left,right,nil,cons
%     literate={\\}{{$\lambda$}}1 {->}{{$\rightarrow$~}}2
%              {<=}{{$\leq$~}}2 {<}{{$<$~}}1
% %     literate={map}{map~}4
% %       {even}{even~}5
% %       {odd}{odd~}4
%      }%
% }
% \defHaskelllistings

\title[SN Guarded Types]{%
  % A Formalized Proof of
Strong Normalization for Guarded Recursive Types}

\author[Abel Vezzosi]{
  \underline{Andreas Abel}
  \and Andrea Vezzosi
}
% \author{Andreas Abel\inst{1}
%   \and Brigitte Pientka\inst{2}
%   \and David Thibodeau\inst{2}
%   \and Anton Setzer\inst{3}
% }
%{F.~Author\inst{1} \and S.~Another\inst{2}}
% - Give the names in the same order as the appear in the paper.
% - Use the \inst{?} command only if the authors have different
%   affiliation.

\institute[Chalmers/GU] % (optional, but mostly needed)
{
  Department of Computer Science and Engineering\\
  Chalmers and Gothenburg University, Sweden \\[1ex]
}
%  \inst{1}%
%  Department of Computer Science\\
%  University of Somewhere
%  \and
%  \inst{2}%
%  Department of Theoretical Philosophy\\
%  University of Elsewhere}
%% - Use the \inst command only if there are several affiliations.
%% - Keep it simple, no one is interested in your street address.

\date[IOC 2014] % (optional, should be abbreviation of conference name)
{ Theory Seminar \\
  Institute of Cybernetics, Tallinn, Estonia \\
  18 December 2014}
% - Either use conference name or its abbreviation.
% - Not really informative to the audience, more for people (including
%   yourself) who are reading the slides online

%\subject{Software Verification}
% This is only inserted into the PDF information catalog. Can be left
% out.



% If you have a file called "university-logo-filename.xxx", where xxx
% is a graphic format that can be processed by latex or pdflatex,
% resp., then you can add a logo as follows:

% \pgfdeclareimage[height=0.5cm]{university-logo}{university-logo-filename}
% \logo{\pgfuseimage{university-logo}}



% Delete this, if you do not want the table of contents to pop up at
% the beginning of each subsection:

%\AtBeginSubsection[]
%\AtBeginSection[]
%{
%  \begin{frame}<beamer>
%    \frametitle{Outline}
%    \tableofcontents[currentsection,currentsubsection]
%  \end{frame}
%}


% If you wish to uncover everything in a step-wise fashion, uncomment
% the following command:

%\beamerdefaultoverlayspecification{<+->}

%\newcommand{\oford}{\of\tord} \newcommand{\oftype}{\of\ttype}
\newcommand{\oford}{}         \newcommand{\oftype}{}
\newenvironment{prg}{\begin{quotation}\begin{tabbing}}{\end{tabbing}\end{quotation}}

% COLORS
\newcommand{\cHead}{\color{darkblue}}
\newcommand{\cSub}{\color{brown}}
\newcommand{\cWhite}{\color{white}}
\newcommand{\cGray}{\color{gray}}
\newcommand{\cGreen}{\color{olivegreen}}
\newcommand{\cBrown}{\color{brown}}
\newcommand{\cBlack}{\color{black}}
\newcommand{\black}[1]{{\cBlack#1}}

\newcommand{\cAnn}{\color{red!80!black}}%purple darkblue
\newcommand{\cAside}{\color{gray}}
\newcommand{\cEnum}{\color{darkgreen}}
\newcommand{\cEm}{\cAnn} %\color{red}}
\newcommand{\cCo}{\cAnn} %\color{red}} % copattern color
\newcommand{\cop}[1]{{\cCo#1}}
\newcommand{\cApp}{\color{violet}}
\newcommand{\capp}[1]{{\cApp#1}}
\newcommand{\cFocus}{\color{darkgreen}}
\newcommand{\focus}[1]{{\cFocus#1}}
\newcommand{\cMath}{\usebeamercolor[fg]{math text}}
\newcommand{\cIdent}{\usebeamercolor[fg]{math text}}
\newcommand{\ident}[1]{{\cIdent#1}}
\newcommand{\cExp}{\cIdent}
\newcommand{\cBoring}{\color{grey}}
\newcommand{\boring}[1]{{\cBoring#1}}

\newcommand{\ann}[1]{^{\cAnn #1}}
\newcommand{\unn}[1]{_{\cAnn #1}}
\newcommand{\annW}[1]{^{\hphantom{#1}}}
%\newcommand{\Ann}[1]{{\cAnn #1}}
\newcommand{\AnnW}[1]{\hphantom{#1}}
\newcommand{\ttAnn}[1]{\{{\cAnn #1}\}}
% ordinal annotation
\newcommand{\cOrd}{\cAnn}
\newcommand{\onn}[1]{^{\cOrd #1}}
\newcommand{\Onn}[1]{{\cOrd #1}}
\newcommand{\oforall}[1]{\forall\Onn{#1}.~}
\newcommand{\oexists}[1]{\exists\Onn{#1}.~}
\newcommand{\oapp}[1]{\,\Onn{#1}}
\renewcommand{\emph}[1]{{\cAnn#1}}
\newcommand{\OSize}{\Onn{Size}}
\newcommand{\oi}{\Onn{i}}
\newcommand{\odi}{\Onn{\$i}}
\newcommand{\oddi}{\Onn{\$\$i}}
\newcommand{\oj}{\Onn{j}}
\newcommand{\ohash}{\Onn{\#}}

\renewcommand{\rulename}[1]{#1}

% types
\newcommand{\cType}{\color{orange!60!black}}
\newcommand{\muT}[2]{\mu {#1} \hspace{-0.1em} . \,  {#2}}

\usepackage{diagrams}
\newarrow{Otto}<--->

\begin{document}
\newcommand{\lat}[1]{\later \! {#1}}
\newcommand{\arr}{\!\to\!}
\newcommand{\lam}[2]{\lambda {#1} \hspace{-0.1em} . {#2}}
\maketitle
%\begin{frame}
%  \titlepage
%\end{frame}

%\begin{frame}
%  \frametitle{Outline}
%  \tableofcontents
%  % You might wish to add the option [pausesections]
%\end{frame}


\section{Introduction}


\begin{frame}%[fragile=singleslide]
  \frametitle{Introduction}
  \begin{itemize}
  \item Guarded recursive types (Nakano, LICS 2000)
  \item Negative recursive types while maintaining consistency
    \begin{itemize}
      \item $\muT X {\lat X \arr  A}$
      \item $\tfix : (\lat A \to A) \to A$
    \end{itemize}
  \item Applications
    \begin{itemize}
      \item Semantics (abstracting step-indexing)
      \item Functional Reactive Programming (causality)
      \item Coinduction (productivity, with a ``Globally''/``$\square$'' modality)
    \end{itemize}
  \vspace{10pt}
  \item \Large This talk: Strong Normalization.
  \end{itemize}
\end{frame}

\begin{frame}%[fragile=singleslide]
  \frametitle{Guarded types}
  \begin{itemize}
  \item Types and terms.
\[
\begin{array}{lrl}
  A,B & ::= &  A \to B \mid \later A \mid X \mid \muT X A
\\
  t,u & ::= & x \mid \lam x t \mid t \, u \mid \nex t \mid t \ast u
\end{array}
\vspace{-1.5ex}
\]
  \item Occurrences of $X$ in $\muT X A$ must be
    under a $\tlater$ ``\alert{guard}''.
  \item Good:
    \begin{itemize}
    \item $\muT X {\later X}$
    \item $\muT X {A \times \later X}$ and $\muT X {\later (A \times X)}$
    \item $\muT X {(\later X) \to A}$ and $\muT X {\later (X \to A)}$.
    \end{itemize}
  \item Bad:
    \begin{itemize}
    \item $\muT X X$ and $\muT X {A \times X}$
    \item $\muT X {X \to A}$ and $\muT X {X \to \later A}$
    \item $\muT X {\later \muT X X}$.
    \end{itemize}
  \end{itemize}
\end{frame}

\begin{frame}%[fragile=singleslide]
  \frametitle{Typing}
  \begin{itemize}
  \item Type equality: congruence closure of $\der \muT X A = \subst
    {\muT X A} X A$.
  \item Typing $\Gamma \der t : A$.
\begin{gather*}
   \ru{\Gamma \der t : A}{\Gamma \der \nex t : \later A}
\qquad
   \ru{\Gamma \der t : \later(A \to B) \qquad
       \Gamma \der u : \later A
     }{\Gamma \der t \ast u : \later B}
\\[2ex]
%   \ru{\Gamma \der t : A}{\Gamma \der t : \top}
% \qquad
  \ru{\Gamma \der t : A \qquad \der A = B
    }{\Gamma \der t : B}
\end{gather*}
  \end{itemize}
\end{frame}

% \begin{frame}%[fragile=singleslide]
%   \frametitle{Denotational Semantics}
%   Types as streams of sets: \\
%   $ A : \mathbb{N} \to Set $ with restriction maps. \\
%   \vspace{10pt}
%   \begin{diagram}
%     A &  A_0 & \lTo^{r_0} & A_1 & \lTo^{r_1} & A_2 & \lTo^{r_2} & \ldots \\
%     \lat A & 1 & \lTo^{!} & A_0 & \lTo^{r_0} & A_1 & \lTo^{r_1} & \ldots
%   \end{diagram}
% \end{frame}


\begin{frame}%[fragile=singleslide]
  \frametitle{Denotational Semantics}
  Types as streams of sets: \\
  $ A : \mathbb{N} \to Set $ with restriction maps. \\
  \vspace{10pt}
  \begin{diagram}
    A &  A_0 & \lTo^{r_0} & A_1 & \lTo^{r_1} & A_2 & \lTo^{r_2} & \ldots \\
    \dTo^{\nex} & \dTo_{!} & & \dTo_{r_0} & & \dTo_{r_1} & & \ldots \\
    \lat A & 1 & \lTo^{!} & A_0 & \lTo^{r_0} & A_1 & \lTo^{r_1} & \ldots
  \end{diagram}
\end{frame}


\begin{frame}%[fragile=singleslide]
  \frametitle{Fixed-point construction (intuition)}
  \vspace{-6ex}
  \begin{diagram}
    \lat A & 1 & \lTo^{!} & A_0 & \lTo^{r_0} & A_1 & \lTo^{r_1} & \ldots \\
    \dTo^f & \dTo_{f_0} & & \dTo_{f_1} & & \dTo_{f_2} & & \ldots \\
    A &  A_0 & \lTo^{r_0} & A_1 & \lTo^{r_1} & A_2 & \lTo^{r_2} & \ldots \\
  \end{diagram}
  Any map $f : \later A \to A$ has a fixed-point $\tfix_f : 1 \to A$:
  %\vspace{10pt}
  \begin{diagram}
    1 & 1  & \rOtto^\tid & 1 & \rOtto^\tid & 1 & \rOtto^\tid & \ldots \\
%                & \dTo^{f_0}\uTo_{!} \\
    \dTo^{\tfix_f} & \dTo_{f_0} &
                & \dTo_{f_1 \circ f_0} &
                & \dTo_{f_2 \circ f_1 \circ f_0} \\
      A & A_0 & \pile{\lTo^{r_0} \\ \rTo_{f_1}}
                & A_1 & \pile{\lTo^{r_1} \\ \rTo_{f_2}}
\               & A_2 & \pile{\lTo^{r_2} \\ \rTo_{f_3}} & \ldots \\
  \end{diagram}

\end{frame}


\begin{frame}%[fragile=singleslide]
  \frametitle{Reduction}
  \begin{itemize}
  \item Redex contraction $t \contr t'$.
\[
\begin{array}{lcl}
  (\lam x t)\, u & \contr & \subst u x t \\
  \nex t \ast \nex u & \contr & \nex (t\,u) \\
\end{array}
\]
  \item Full one-step reduction $t \red t'$: Compatible closure of $\contr$.
  \end{itemize}
\end{frame}


\begin{frame}%[fragile=singleslide]
  \frametitle{Recursion from recursive types}
Guarded recursion combinator can be encoded. \\
The standard $\Y$ combinator would need a type $T$ such that
\[ T = T \to A \]
to typecheck the self applications of $x$ and $\omega$:\\
\[
\begin{array}{lllllll}
  &&f & : & A \to A \\
  &&x\,x & : & A & \mif x : T \\
  %% &&B & := & \muT X {\lat X \to A} & = & \lat B \to A \\
  %% &&x & : & \lat B & = & \lat (\lat B \to A) \\
  %% &&x \ast \nex x & : & \lat A \\
  %% &&f\,(x \ast \nex x) & : & A \\
  %% \omega & := & \lambda x. \, f\,(x \ast \nex x) & : & \lat B \to A  & = & B \\
  %% \Y     & := & \omega \, (\nex \omega) & : & A \\
  \omega & := & \lambda (x : T). \, f\,(x \, x) & : & T \to A \\
  \Y     & := & \omega \, \omega & : & A \\
\end{array}
\]
\end{frame}

\begin{frame}%[fragile=singleslide]
  \frametitle{Recursion from recursive types}

We can solve $T = \lat T \to A$:
\[
T = \muT X {\lat X \to A}
\]
So we get a guarded fixpoint combinator:
\[
\begin{array}{lllllll}
  &&f & : & \lat A \to A \\
  && x & : & \lat (\lat T \to A) & \mif{ x : \lat T} \\
  && x \ast \nex x & : & \lat A & \mif{ x : \lat T} \\
  \omega & := & \lambda (x : \lat T). \, f\,(x \ast \nex x) & : & \lat T \to A \\
  \Y_f     & := & \omega \, (\nex \omega) & : & A \\
\\
\multicolumn 7 l {
  \Y_f \red f\,(\nex \omega \ast \nex (\nex \omega)) \red
  f\,(\nex (\omega \, (\nex \omega))) = f\, (\nex \Y_f)
}
\end{array}
\]
Note: Full reduction $\red$ of $\Y_f$ diverges.
\end{frame}


\begin{frame}%[fragile=singleslide]
  \frametitle{More Examples}
  \begin{itemize}
  \item Streams!?
  \item RepMin: One pass through binary tree, replacing all labels by their
    minimum.
  \item Attribute grammars!?
  \end{itemize}
\end{frame}



\begin{frame}%[fragile=singleslide]
  \frametitle{Restricted reduction}
  \begin{itemize}
  \item Restore normalization: do not reduce under $\tnext$.
  \item Relaxed: reduce only under $\tnext$ up to a certain depth.
  \item Family $\red_n$ of reduction relations.
\[
  \ru{t \contr t'}{t \red_n t'}
\qquad
  \ru{t \red_n t'}{\nex t \red_{n+1} \nex t'}
\]
  \item Plus compatibility rules for all other term constructors.
  \item $\red_n$ is monotone in $n$  (more fuel gets you further).
  \item Goal: each $\red_n$ is strongly normalizing.
  \end{itemize}
\end{frame}

\begin{frame}%[fragile=singleslide]
  \frametitle{Restricted reduction (Example)}
\[
  \begin{array}{l}
    \Y \red^*_0 f\, (\nex \Y) \longarrownot\longrightarrow_0 \\ \\
    \Y \red^*_1 f\, (\nex (f\, (\nex \Y))) \longarrownot\longrightarrow_1   \\ \\
    \Y \red^*_2 f\, (\nex (f\, (\nex (f\, (\nex \Y))))) \longarrownot\longrightarrow_2   \\ \\
    \vdots
  \end{array}
\]
\end{frame}


\begin{frame}%[fragile=singleslide]
  \frametitle{Strong normalization as well-foundedness}
  \begin{itemize}
  \item $t \in \sn_n$ if $\red_n$ reduction starting with $t$
    terminates.
\[
  \ru{\forall t'.\, t \red_n t' \implies t' \in \sn_n
    }{t \in \sn_n}
\]
  \item $\sn_n$ is antitone in $n$, since $\red_n$ occurs negatively.
  \item More reductions $\implies$ less termination.
  \end{itemize}
\end{frame}


\begin{frame}
  \frametitle{Inductive $\SN_n$}
  \vspace{-15pt}
  \begin{itemize}
  \item Take the inductively defined normal forms:
\begin{gather*}
  E ::= \_ \mid E\,u \mid E \ast u \mid \nex t \ast E\\
  \ru{E \in \SN_n
    }{E[x] \in \SN_n}
\qquad
  \ru{t\in\SN_n}{\lam x t \in \SN_n}
\qquad
  \ru{}{\nex t \in \SN_0}
\qquad
  \ru{t \in \SN_n
    }{\nex t \in \SN_{n+1}}
\end{gather*}
\item And close them under ``Strong head reduction'' $t \redSN_n t'$
\begin{gather*}
\ru{t \redSN_n t' \qquad t' \in \SN_n }
   {t \in \SN_n}\qquad
\ru{t \contr t' \qquad t \in \SN_n}
   {E[t] \redSN_n E[t']}
\end{gather*}
\item $t \redSN_n t'$ is like weak head reduction but erased terms must be s.n.
\end{itemize}
\end{frame}

\begin{frame}%[fragile=singleslide] %% TODO cut?
  \frametitle{Notions of s.n.\ coincide?}
  \begin{itemize}
  \item Rules for $\SN_n$ are closure properties of $\sn_n$.
  \item $\SN_n \subseteq \sn_n$ follows by induction on $\SN_n$.
  \item Converse $\sn_n \subseteq \SN_n$ does not hold!
  \item Counterexamples are ill-typed s.n. terms, e.g.,
\[
  (\lambda x.\, x) \ast y
\qquad \mybox{or}\qquad
  (\nex x)\,y
.\]
  \item Solution: consider only well-typed terms.
  \item Proof of $t \in \sn_n \implies t \in \SN_n$ by case
    distinction on $t$: neutral ($E[x]$), introduction ($\lam
    x t, \nex t$), or weak head redex.
  \end{itemize}
\end{frame}


\begin{frame}%[fragile=singleslide]
  \frametitle{Saturated sets (semantic types)}
  \begin{itemize}
  \item Types are modeled by sets $\A \subseteq \SN_n$.
%%  \item Semantic function space should contain $\lambda$s and terms
%%    that weak head reduce to $\lambda$s. % TODO ask Andreas
  \item $n$-closure $\clos \A _n$ of $\A$ inductively:
\[
  \ru{t \in \A}{t \in \clos \A _n}
\qquad
  \ru{E \in \SN_n}{E[x] \in \clos \A _n}
\qquad
  \ru{t \redSN_n t' \qquad t' \in \clos \A _n
    }{t \in \clos \A _n}
\]
  \item $\A$ is $n$-saturated ($\A \in \SAT_n$) if $\clos\A_n
    \subseteq \A$.
  \item Saturated sets are non-empty (contain e.g. the variables).
  \end{itemize}
\end{frame}


\begin{frame}%[fragile=singleslide]
  \frametitle{Constructions on semantic types}
  \begin{itemize}
  \item Function space and ``later'':
\[
\begin{array}{lcl}
  \A \to \B   & = & \{ t \mid t\,u \in \B \mforall u \in \A \} \\[1ex]
  \slater n \A & = & \clos{\{ \nex t \mid t \in \A \mif n>0 \}}_n \\
\end{array}
\]
  \item If $\A,\B \in \SAT_n$ then $\A \to \B \in \SAT_n$.
  \item $\slater 0 \A \in \SAT_0$.
  \item If $\A \in \SAT_n$ then $\slater {n+1} \A \in \SAT_{n+1}$.
  \end{itemize}
\end{frame}


\begin{frame}%[fragile=singleslide]
  \frametitle{Type interpretation}
  \begin{itemize}
  \item Type interpretation $\Den A n \in \SAT_n$
\[
\def\arraystretch{1.3}
 \begin{array}{lcl}
   \Den{A \to B}n & = & \bigcap_{n' \leq n} (\Den A{n'} \to \Den{B}{n'})
   \\
   \Den{\later A}0 & = & \slater 0 \SN_0 = \clos{\{ \nex t \}}_0 \\
   \Den{\later A}{n+1} & = & \slater{n+1} \Den A n \\
   \Den{\muT X A}n & = & \Den{\subst {\muT X A} X A}n \\
 \end{array}
\]
\vspace{-2ex}
\item
By lex.\ induction on $(n,\tsize(A))$ where $\tsize(\later A) = 0$.
\item Requires recursive occurrences of $X$ to be \emph{guarded} by a $\tlater$.
  \end{itemize}
\end{frame}


\begin{frame}%[fragile=singleslide]
  \frametitle{Type soundness}
  \begin{itemize}
  \item Context interpretation:
\[
  \rho \in \Den \Gamma n \iff \rho(x) \in \Den{A}n \mforall
  (x{:}A) \in \Gamma
\]
\vspace{-2ex}
  \item Identity substitution $\tid \in \Den \Gamma n$ since $x \in \Den A n$.
  \item Type soundness: if $\Gamma \der t : A$ then $t \rho \in \Den A
    n$ for all $n$ and $\rho \in \Den \Gamma n$.
  \item Corollary: $t \in \SN_n$ for all $n$.
  \end{itemize}
\end{frame}


\begin{frame}%[fragile=singleslide]
  \frametitle{Formalization in Agda}
  \vspace{-10pt}
  Syntax of types as a mixed inductive-coinductive datatype:
  \vspace{-5pt}
  \[\mathsf{Ty} = \nu X \mu Y.\; (Y \times Y) + X \]
  \vspace{-30pt}
  \input{TypesTalk}
  \begin{itemize}
  \item Intensional (propositional) equality too weak for coinductive
    types.
  \item $\implies$ add an extensionality axiom for our coinductive type.
  \end{itemize}
\end{frame}


\begin{frame}%[fragile=singleslide]
  \frametitle{Well-typed terms}
  \input{TermsTalk}
  \begin{itemize}
  \item We used intrinsically well-typed terms (data structure indexed
    by typing context and type expression).
  \item Second Kripke dimension (context) required ``everywhere'', e.g.,
    in $\SN$ and $\den A$.
  \end{itemize}
\end{frame}




\begin{frame}%[fragile=singleslide]
  \frametitle{Conclusions \& Further work}
  \begin{itemize}
  \item \emph{Strong} normalization is a new result, albeit expected for the restricted reduction.
  \item Agda formalization (ca. 3kLoc, 170kB) useful as basis for further research.
  \item Add modalities to handle (co)inductive types.
  \item Integrate into Intensional Type Theory.
  \end{itemize}
\end{frame}



%% \begin{frame}%[fragile=singleslide]
%%   \frametitle{Acknowledgments}
%%   \begin{itemize}
%%   \item Rasmus M\o{}gelberg, for discussions and the present invitation.
%%   \item Lars Birkedal, for a previous invitation that initiated this
%%     research.
%%   \item Neel Krishnaswami, for email input.
%%   \item The (other) Agda developers, especially Stevan Andjelkovic for
%%     the \LaTeX backend.
%%   \end{itemize}
%% \end{frame}

%%%%%%%%%%%%%%%%%%%%%%%%%%%%%%%%%%%%%%%%%%%%%%%%%%%%%%%%%%%%%%%%%%%%%%
%%%%%%%%%%%%%%%%%%%%%%%%%%%%%% END DOC %%%%%%%%%%%%%%%%%%%%%%%%%%%%%%%
%%%%%%%%%%%%%%%%%%%%%%%%%%%%%%%%%%%%%%%%%%%%%%%%%%%%%%%%%%%%%%%%%%%%%%

% \bibliography{short}

\end{document}



\begin{frame}%[fragile=singleslide]
  \frametitle{}
  \begin{itemize}
  \item
  \end{itemize}
\end{frame}


\begin{frame}%[fragile=singleslide]
  \frametitle{}
  \begin{itemize}
  \item
  \end{itemize}
\end{frame}


\begin{frame}%[fragile=singleslide]
  \frametitle{}
  \begin{itemize}
  \item
  \end{itemize}
\end{frame}


\begin{frame}%[fragile=singleslide]
  \frametitle{}
  \begin{itemize}
  \item
  \end{itemize}
\end{frame}


\begin{frame}%[fragile=singleslide]
  \frametitle{}
  \begin{itemize}
  \item
  \end{itemize}
\end{frame}


\begin{frame}%[fragile=singleslide]
  \frametitle{}
  \begin{itemize}
  \item
  \end{itemize}
\end{frame}


\begin{frame}%[fragile=singleslide]
  \frametitle{}
  \begin{itemize}
  \item
  \end{itemize}
\end{frame}





%%% Local Variables:
%%% mode: latex
%%% TeX-master: t
%%% End:
