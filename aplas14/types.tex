
\section{Guarded Recursive Types and Their Semantics}
\label{sec:types}

Nakano's type system \citeyearpar{nakano:lics00} is equipped with subtyping,
but we stick to a simpler variant without, a simply-typed version of
\citet{birkedalMogelberg:lics13}, which we shall call $\lambdalater$.  
Our rather minimal grammar of types
includes product $A \times B$ and function types $A \to B$, delayed
computations $\later A$, variables $X$ and explicit fixed-points $\mu
X A$.
\[
\begin{array}{lrl@{\qquad}l}
  A, B, C & ::= & A \times B \mid A \to B \mid \later A \mid X \mid
    \mu X A
\end{array}
\]
Base types and disjoint sum types could be added, but would only give
breadth rather than depth to our formalization.  As usual, a dot after
a bound variable shall denote an opening parenthesis that closes as far
to the right as syntactically possible.  Thus, $\mu X.\, X \to X$
denotes $\mu X \,(X \to X)$, while $\mu X X \to X$ denotes $(\mu X.X)
\to X$ (with a free variable $X$).

Formation of fixed-points $\mu X A$ is subject to the side condition that $X$ is
guarded in $A$, \ie, $X$
appears in $A$ only under a \emph{later} modality $\tlater$.  
This rules out all unguarded recursive types like $\mu X.\, A \times
X$ or $\mu X.\, X \to A$, but allows their variants 
$\mu X.\, \later (A \times X)$ and  $\mu X.\, A \times \later X$,
and $\mu X.\, \later (X \to A)$ and $\mu X.\, \later X \to A$. 
Further, fixed-points give rise to an equality relation on types
induced by $\mu X A = \subst {\mu X A} X A$.

\begin{figure}[htbp]
\begin{gather*}
\ru{\Gamma(x) = A
  }{\Gamma \der x : A} 
\qquad
\ru{\Gamma, x \of A \der t : B
  }{\Gamma \der \lambda x . \; t : A \to B}
\qquad
\ru{\Gamma \der t : A \to B \qquad
    \Gamma \der u : A
  }{\Gamma \der t\,u : B}
\\[2ex]
\ru{\Gamma \der t_1 : A_1 \qquad 
    \Gamma \der t_2 : A_2 
  }{\Gamma \der (t_1, t_2) : A_1 \times A_2}
\qquad
\ru{\Gamma \der t : A_1 \times A_2 
  }{\Gamma \der \fst t : A_1}
\qquad  
\ru{\Gamma \der t : A_1 \times A_2 
  }{\Gamma \der \snd t : A_2}
\\[2ex]
\ru{\Gamma \der t : A
  }{\Gamma \der \pure t : \later A}
\qquad
\ru{\Gamma \der t : \later(A \to B) \qquad
    \Gamma \der u : \later A
  }{\Gamma \der \apply t u : \later B}
\qquad
\ru{\Gamma \der t : A \qquad
    A = B
  }{\Gamma \der t : B} 
\end{gather*}
\caption{Typing rules.}
\label{fig:typing}
\end{figure}

Terms are lambda-terms with pairing and projection plus operations
that witness \emph{applicative functoriality} of the later modality
\citep{atkeyMcBride:icfp13}. 
\[
\begin{array}{lrl@{\qquad}l}
  r,s,t,u & ::= & x \mid \lambda x \, t \mid t\,u \mid (t_1,t_2) \mid
  \fst t \mid \snd t \mid \pure t \mid \apply t u
\end{array}
\]
Figure~\ref{fig:typing} recapitulates the static semantics.  The
dynamic semantics are induced
by the following \emph{contractions}:
\[
\begin{array}{l@{~~}l@{~~}l}
  (\lambda x . \; t)\,u          & \contr & \subst u x t \\
  \fst(t_1,t_2)             & \contr & t_1 \\
  \snd(t_1,t_2)             & \contr & t_2 \\
  \applyp{\pure t}{\pure u} & \contr & \pure(t\,u) \\
\end{array}
\]
If we conceive our small-step reduction relation $\red$ as 
the compatible closure of $\contr$, we obtain a non-normalizing
calculus, since terms like $\Omega = \omega \; (\pure\omega)$ with
$\omega = (\lambda x.\; \apply x (\pure x))$ are
typeable.\footnote{$\der \Omega : A$ with $A = \mu X (\later X)$.  To type
  $\omega$, we use $x : \mu Y (\later (Y \to A))$.} 
Unrestricted reduction of $\Omega$ is non-terminating:
$\Omega \red \pure \Omega \red \pure (\pure \Omega) \red \dots$
If we let $\tnext$ act as delay operator that blocks reduction inside,
we regain termination.  In general, we preserve termination if we only
look under delay operators until a certain depth.  This can be made
precise by a family $\red_n$ of reduction relations indexed by depth
$n \in \NN$, see Figure~\ref{fig:red}
\begin{figure}[htbp]
\begin{gather*}
\ru{t \contr t'
  }{t \red_n t'}
\qquad
\ru{t \red_n t'
  }{\lambda x. \; t \red_n \lambda x. \; t'}
\qquad
\ru{t \red_n t' 
  }{t\,u \red_n t'\,u}
\qquad
\ru{u \red_n u'
  }{t\,u \red_n t\,u'}
\\[2ex]
\ru{t \red_n t' 
  }{(t,u) \red_n (t',u)}
\qquad
\ru{u \red_n u'
  }{(t,u) \red_n (t,u')}
\qquad
\ru{t \red_n t'
  }{\fst t \red_n \fst t'}
\qquad
\ru{t \red_n t'
  }{\snd t \red_n \snd t'}
\\[2ex]
\fbox{$
\ru{t \red_n t'
  }{\pure t \red_{n+1} \pure t'}
$}
\qquad
\ru{t \red_n t' 
  }{\apply t u \red_n \apply {t'} u}
\qquad
\ru{u \red_n u'
  }{\apply t u \red_n \apply t u'}
\end{gather*}
\caption{Reduction}
\label{fig:red}
\end{figure}


To show termination, we interpret types as sets $\A,\B,\C$ of
depth-$n$ strongly normalizing terms. 
We define semantic versions $\semtimes$,
$\semto$, and $\semlater$ of product, function space, and delay type
constructor, plus a terminal (=largest) semantic type $\semtop$.  Then
the interpretation $\Den A n$ of closed type $A$ at depth $n$ can be given
recursively as follows, using Kripke-style at function types:
\[
\def\arraystretch{1.3}
\begin{array}{lll}
  \Den{A \times B}n & = & \Den A n \semtimes \Den B n \\
  \Den{A \to B}n & = & \bigcap_{n' \leq n} (\Den A {n'} \semto \Den B
  {n'}) \\
  \Den{\later A}0 & = & \semtop \\
  \Den{\later A}{n+1} & = & \semlater \Den A n \\
  \Den{\mu X A}n & = & \Den{\subst{\mu X A}X A}n \\
\end{array}
\]
% The \emph{stage} $n$ can be seen as \emph{time} or evaluation
% \emph{depth}.  At function types, we employ a Kripke semantics.
%
Due to the last equation $(\mu)$, the type interpretation is ill-defined for
unguarded recursive types.  However, for guarded types we only return
to the fixed-point case after we have passed the case for $\later$,
which decreases the index $n$.  More precisely, $\Den A n$ is defined by
lexicographic induction on $(n, \tsize(A))$, where $\tsize(A)$ is the
number of type constructor symbols ($\times$, $\to$, $\mu$)
that occur \emph{unguarded} in $A$.

While all this sounds straightforward at an informal level, formalization
of the described type language is quite hairy.  For one, we have to
enforce the restriction to well-formed (guarded) types.  Secondly, our
type system contains a conversion rule, getting us into the vincinity
of dependent types which are still a challenge to a completely formal
treatment \citep{mcBride:wgp10}.  Our first formalization attempt used
kinding rules for types to keep track of guardedness for formation of
fixed-point, and a type equality relation, and building on this,
inductively defined well-typed terms.  However, the complexity was
discouraging and lead us to a much more economic representation of
types, which is described in the next section.

%%% Local Variables: 
%%% mode: latex
%%% TeX-master: "aplas14.tex"
%%% End: 
