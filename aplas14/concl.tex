
\section{Conclusions}
\label{sec:concl}

In this paper, we presented a strongly-normalizing reduction relation
for simply-typed lambda calculus with Nakano's modality for
recursion.  Correctness has formally been proven in Agda using a
saturated sets semantics based on an inductive notion of strong
normalization.  Herein, we represented recursive types as infinite
type expressions and terms as intrinsically well-typed ones.  

Our treatment of infinite type expressions was greatly simplified by
adding an extensionality axiom for the underlying coinductive type to
Agda's type theory.  This would not have been necessary in a more
extensional theory such as \emph{Observational Type Theory}
\citep{altenkirchMcBrideSwierstra:plpv07} as shown in \citep{mcBride:unfold}.  Possibly \emph{Homotopy Type
Theory} \citep{hott} 
would also address this problem, but there the status of
coinductive types is yet unclear.

For the future, we would like to investigate how to incorporate
guarded recursive types into a dependently-typed language, and how
they relate to other approaches like coinduction with sized
types, for instance.

\paradot{Acknowledgments}
Thanks to Lars Birkedal, Ranald Clouston, and Rasmus M\o{}gelberg for
fruitful discussions on guarded recursive types, and Hans Bugge
Grathwohl and Fabien Renaud for useful feedback on the Agda
development and a draft version of this paper.
The first author acknowledges support by Vetenskapsr\aa{}det framework
grant 254820104 (Thierry Coquand).  This paper has been prepared with
Stevan Andjelkovic's Agda-to-LaTeX converter. 


%%% Local Variables: 
%%% mode: latex
%%% TeX-master: "aplas14.tex"
%%% End: 
