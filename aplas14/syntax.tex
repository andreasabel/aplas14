
\section{Formalized Syntax}
\label{sec:syntax}

In this section, we discuss the formalization of types, terms, and
typing of $\lambdalater$ in Agda.  It will be necessary to talk about
meta-level types, \ie, Agda's types, thus, we will refer to
$\lambdalater$'s type constructors as $\hattimes$, $\hatto$,
$\hatlater$, and $\hatmu$.

\subsection{Types Represented Coinductively}

Instead of representing fixed-points as syntactic construction on
types, which would require a non-trivial equality on types induced by
$\hatmu X A = \subst{\hatmu X A}X A$, we use \emph{meta-level} fixed-points,
\ie, Agda's recursion mechanism.  \footnote{
An alternative to get around the type equality problem would be
iso-recursive types, \ie, with term constructors for folding and
unfolding of $\hatmu X A$.  However, we would still have to implement
type variables, binding of type variables, type substitution, lemmas
about type substitution etc.
}
Extensionally, we are implementing
\emph{infinite type expressions} over the constructors $\hattimes$,
$\hatto$, and $\hatlater$.  The guard condition on recursive types then
becomes an instance of Agda's ``guard condition'', \ie, the condition
the termination checker imposes on recursive programs.

Viewed as infinite expressions, guarded types are regular trees with
an infinite number of $\hatlater$-nodes on each infinite path.  This
can be expressed as the mixed coinductive-inductive (meta-level) type
\[
  \nu X \mu Y.\; (Y \times Y) + (Y \times Y) + X.
\]
The first summand stands for the binary constructor $\hattimes$, the
second for $\hatto$, and the third for the unary $\hatlater$.  The
nesting of a least-fixed point ($\mu$) inside a greatest fixed-point
($\nu$) ensures that on each path, we can only take alternatives
$\hattimes$ and $\hatto$ a finite number of times before we have to
choose the third alternative $\hatlater$ and restart the process.


\input{InfiniteTypes}

\subsection{Well-typed terms}

Instead of a raw syntax and a typing relation, we represent well-typed
terms directly by an inductive family
\citep{dybjer:inductiveFamilies}.
Our main motivation for this
choice is the beautiful inductive definition of strongly normalizing terms to
follow in Section~\ref{sec:sn}.  Since it relies on a classification
of terms into the three shapes \emph{introduction},
\emph{elimination}, and \emph{weak head redex}, it does not capture all strongly normalizing raw terms, in particular
``junk'' terms such as $\fst (\lambda x x)$.  Of
course, statically well-typed terms come also at cost: for almost all
our predicates on terms we need to show that they are natural in the
typing context, \ie, closed under well-typed renamings.  This expense
might be compensated by the extra assistance Agda can give us in
proof construction, which is due to the strong constraints on possible
solutions imposed by the rich typing.

\input{Terms}


%%% Local Variables:
%%% mode: latex
%%% TeX-master: "aplas14.tex"
%%% End:
