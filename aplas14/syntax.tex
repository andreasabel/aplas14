
\section{Formalized Syntax}
\label{sec:syntax}

In this section, we discuss the formalization of types, terms, and
typing of $\lambdalater$ in Agda.  It will be necessary to talk about
meta-level types, \ie, Agda's types, thus, we will refer to
$\lambdalater$'s type constructors as $\hattimes$, $\hatto$,
$\hatlater$, and $\hatmu$.

\subsection{Types Represented Coinductively}

Instead of representing fixed-points as syntactic construction on
types, which would require a non-trivial equality on types induced by
$\hatmu X A = \subst{\hatmu X A}X A$, we use \emph{meta-level} fixed-points,
\ie, Agda's recursion mechanism.  Extensionally, we are implementing
\emph{infinite type expressions} over the constructors $\hattimes$,
$\hatto$, and $\hatlater$.  The guard condition on recursive types then
becomes an instance of Agda's ``guard condition'', \ie, the condition
the termination checker imposes on recursive programs.

Viewed as infinite expressions, guarded types are regular trees with
an infinite number of $\hatlater$-nodes on each infinite path.  This
can be expressed as the mixed coinductive-inductive (meta-level) type
\[
  \nu X \mu Y.\; (Y \times Y) + (Y \times Y) + X. 
\]
The first summand stands for the binary constructor $\hattimes$, the
second for $\hatto$, and the third for the unary $\hatlater$.  The
nesting of a least-fixed point ($\mu$) inside a greatest fixed-point
($\nu$) ensures that on each path, we can only take alternatives
$\hattimes$ and $\hatto$ a finite number of times before we have to
choose the third alternative $\hatlater$ and restart the process.


\input{InfiniteTypes}

\input{Terms}


%%% Local Variables: 
%%% mode: latex
%%% TeX-master: "aplas14.tex"
%%% End: 
