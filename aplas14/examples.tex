
\subsection{Type Equality}
\label{sec:tyeq}

Although our coinductive representation of $\lambdalater$ types saves
us from type variables, type substitution, and fixed-point unrolling,
the question of type equality is not completely settled.  
The propositional equality $\propeq$ of Martin-L\"of Type Theory is intensional
in the sense that only objects with the same \emph{code} (modulo
definitional equality) are considered equal.  Thus, $\propeq$ is
adequate only for finite objects (such as natural numbers and lists)
but not for infinite objects like functions, streams, or
$\lambdalater$ types.

However, we can define extensional equality or \emph{bisimulation} 
on $\Ty$ as a mixed
coinductive-inductive relation $\bisim/\bisiminf$ that follows the
structure of $\Ty/\infTy$ (hence, we reuse the constructor names
$\hattimes$, $\hatto$, and $\hatlater$).
  
\input{TypeEquality}

\subsection{Examples}
\label{sec:examples}

\input{AgdaExamples}

%%% Local Variables: 
%%% mode: latex
%%% TeX-master: "aplas14.tex"
%%% End: 
