
\subsection{Type Equality}
\label{sec:tyeq}

Although our coinductive representation of $\lambdalater$ types saves
us from type variables, type substitution, and fixed-point unrolling,
the question of type equality is not completely settled.
The propositional equality $\propeq$ of Martin-L\"of Type Theory is intensional
in the sense that only objects with the same \emph{code} (modulo
definitional equality) are considered equal.  Thus, $\propeq$ is
adequate only for finite objects (such as natural numbers and lists)
but not for infinite objects like functions, streams, or
$\lambdalater$ types.

However, we can define extensional equality or \emph{bisimulation}
on $\Ty$ as a mixed
coinductive-inductive relation $\bisim/\bisiminf$ that follows the
structure of $\Ty/\infTy$ (hence, we reuse the constructor names
$\hattimes$, $\hatto$, and $\hatlater$).

\input{TypeEquality}

\subsection{Examples}
\label{sec:examples}

Following \cite{nakano:lics00},
we can adapt the $Y$ combinator from the untyped lambda calculus to
define a guarded fixed point combinator:
\[
  \tfix = \lambda f.\; (\lambda x. \; f \; (\apply x {\pure x}))\; (\tnext \; (\lambda x. \; f \; (\apply x {\pure x})))
.\]
We construct an auxiliary type \AgdaFunction{Fix}
\va{} that allows safe self application, since the argument will only
be available "later". This fits with the type we want for the
\AgdaFunction{fix} combinator,
which makes the recursive instance $y$ in $\tfix\;(\lambda y.\, t)$ available only  at the next time slot.
% the function of which we are taking the
% fixed point will only be able to use its input at the next time slot.

\input{AgdaExamples}

%%% Local Variables:
%%% mode: latex
%%% TeX-master: "aplas14.tex"
%%% End:
