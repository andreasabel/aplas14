\section{Strong Normalization}
\label{sec:sn}

\input{sn-definition}

\Citet{raamsdonk:perpetualReductions} pioneered a more explicit
characterization of strongly normalizing terms $\SN$, namely the least set
closed under introductions, formation of neutral (=stuck) terms, and
weak head expansion.  We adapt their technique from lambda-calculus to
$\lambdalater$; herein, it is crucial to work with well-typed terms to
avoid junk like $\tfst\,(\lambda x. \, x)$ which does not exist in pure
lambda-calculus.  To formulate a deterministic weak head evaluation,
we make use of the \emph{evaluation contexts} $\vE : \ECxt$
\[
  \vE ::= \_\;u \mid \fst\_ \mid \snd\_ \mid \_\,∗\,u \mid (\tnext\;t)\,∗\,\_
.\]
Since weak head reduction does not go into introductions which include
$\lambda$-abstraction, it does not go under binders, leaving typing
context \Gam{} fixed.

\input{TermShape}

The family of predicates $\SN\;\vn$
is defined inductively by the following rules---we allow ourselves
set-notation at this semi-formal level:
\begin{gather*}
  \ru{t \in \SN\;\vn
    }{\lambda x t \in \SN\;\vn}
\qquad
  \ru{t_1,t_2 \in \SN\;\vn
    }{(t_1,t_2) \in \SN\;\vn}
\qquad
  \ru{}{\pure t \in \SN\;0}
\qquad
  \ru{t \in \SN\;\vn
    }{\pure t \in \SN\;(1 + \vn)}
\\[2ex]
  \ru{t \in \SNe\;\vn
    }{t \in \SN\;\vn}
\qquad
  \ru{t' \in \SN\;\vn \qquad t \nwhr n t' % t \whr_{\SN\;\vn} t'
    }{t \in \SN\;\vn}
\end{gather*}
The last two rules close $\SN$ under neutrals $\SNe$, which is an
instance of $\PNe$ with $\vP = \SN\;\vn$,
and level-$\vn$
\emph{strong head expansion} $t \nwhr n t'$,
which is an instance of $\vP$-whe with
also $\vP = \SN\;\vn$.
\LONGVERSION{
We represent the inductive $\SN$ in Agda as a sized
type \citep{hughesParetoSabry:popl96,abelPientka:icfp13} for the
purpose of termination checking certain inductions on $\SN$ later.
The assignment of sizes follows the principle that recursive
invokations of $\SN$ within a constructor of $\SN\;\{\vi\}$
must carry a strictly smaller size $\vj : \SizeLt\;\vi$.
The mutually defined relations $\SNe\;\vn\;\vt$ (instance of $\PNe$) and
strong head reduction (shr)
$\vt \nwhr n \vtprime$ just thread the size argument through.  Note
that there is a version $\vi\;\asize\;\vt \nwhr n \vtprime$ of shr
that makes the size argument visible, to be supplied in case $\texp$.
}
\input{SN}

\input{AntiRename}


%%% Local Variables:
%%% mode: latex
%%% TeX-master: "aplas14.tex"
%%% End:
