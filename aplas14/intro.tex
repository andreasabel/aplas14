
\section{Introduction}
\label{sec:intro}

In untyped lambda calculus, fixed-point combinators can be defined
using self-app\-li\-ca\-tion.  Such combinators can be assigned recursive
types, albeit only negative ones.  Since such types introduce logical
inconsistency, they are problematic, especially in Martin-L\"of
Type Theory and other systems based on the Curry-Howard isomorphism.
\citet{nakano:lics00} introduced \emph{a modality for recursion} that
allows a stratification of negative recursive types to recover
consistency.  In essence, each negative recursive occurrence needs to
be \emph{guarded} by the modality; this coined the term \emph{guarded
  recursive types} \citep{birkedalMogelberg:lics13}.\footnote{Not to
  be confused with \emph{Guarded Recursive Datatype Constructors}
  \citep{xiChenChen:popl03}.} 
Nakano's modality has found applications in functional reactive
programming \citep{krishnaswamiBenton:lics11} where it is referred to
as \emph{later} modality.

While Nakano showed that every typed term has a weak head normal form,
in this paper we prove \emph{strong normalization} for our variant
$\lambdalater$ of Nakano's calculus.  
To this end, we
make the introduction rule for the later modality explicit in the
terms by a constructor $\tnext$, following
\citet{birkedalMogelberg:lics13} and \citet{atkeyMcBride:icfp13}.  By
allowing reduction under finitely many $\tnext$s, we establish
termination irrespective of the reduction strategy.  
%
Showing strong normalization of $\lambdalater$ is a first step towards
a operationally well-behaved type theory with guarded recursive types, for
which \citet{birkedalMogelberg:lics13} have given a categorical model.

Our proof is fully formalized in the proof assistant Agda
\citep{agda24} which is based on intensional Martin-L\"of Type Theory.
One key idea of the formalization is to represent the recursive types
of $\lambdalater$ as infinite type expressions in form of a
coinductive definition.  For this, we utilize Agda's new
\emph{copattern} feature \cite{abelPientkaThibodeauSetzer:popl13}.  
The set of strongly normalizing terms is defined inductively by
distinguishing on the shape of terms
\citep{raamsdonk:perpetualReductions,matthes:shortproofs}.  The first author has
formalized this technique before in Twelf \citep{abel:entcs04};
in this work we extend these results by a  
proven relation to the standard notion of strong normalization. 

Due to space constraints, we can only give a sketch of the
formalization; a longer version and
the full Agda proofs 
are available online
\citep{abelVezzosi:guardedNormalization}.
This paper is extracted from a literate Agda file; all the colored
code in displays is necessarily type-correct.

%%% Local Variables: 
%%% mode: latex
%%% TeX-master: "aplas14.tex"
%%% End: 
